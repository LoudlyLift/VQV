\documentclass[11pt,oneside,letterpaper]{article}
\usepackage[left=1in, right=1in, top=1in, bottom=1in]{geometry}

\usepackage[pdftex]{graphicx}
\graphicspath{{./Figures/}}
\DeclareGraphicsExtensions{.pdf,.jpeg,.png}


\usepackage{amsmath,amsthm,amssymb} %math proofs
\usepackage{spalign,systeme,amsmath}
\usepackage{verbatim}
\usepackage{verbatimbox}
\usepackage{indentfirst}
\usepackage[final]{pdfpages}
\usepackage[toc,page]{appendix}
\usepackage[utf8]{inputenc}
\usepackage{hyperref}
\usepackage{float}


\begin{document}

\title{UH 400: Independent Study}

\author{Ivan Johnson}
\date{2019 May 3}

\maketitle

For my independent study project, I decided to create a video encoder that is
capable of encoding different regions of a single video at different quality
levels. This was quite the ambitious goal, as existing tools are not designed
for this, and I had no experience with video file formats prior to this project.

I started by finding online resources to teach myself how video encoders worked,
and found them to be much more complicated than I had initially believed. There
are, however, a number of great resources on the internet that provide easy to
understand explanations of how videos work.

Eventually, I became confident enough in my understanding that I moved on to
trying to start implementing my ideas. It would be completely unrealistic for me
to try to create everything from scratch. As such, I searched for and found a
half dozen or so different encoders online that allow people like me to view and
modify their source code. Eventually, I settled on using one called ``Kvazaar''
that seemed like it would be easy for me to use, as they appeared to have the
best documentation about how it worked.

When I started to use Kvazaar in earnest, however, I found it to be a
complicated beast. The functionality that I needed to modify for my variable
quality video encoder was not coalesced in a single file as I had expected, but
appeared to be duplicated several times across the entire project, each with its
own slight tweaks for the needs in that particular part of the project. This led
to progress towards a working result being very slow.

When the end of the winter 2018 semester came about, I was very unhappy with my
results so Dr.\ Chakareski and I elected to continue the project in the spring
of 2019. Ultimately, we realized that I shouldn't have been so focused on trying
to do things in the best possible way, and instead I should take a much simpler,
albeit somewhat less efficient, approach. In keeping with Hofstadter's law, this
new approach took much longer to finish than initially expected. However, I
finally managed to get it working at the end of the semester.

The complete set of files I've been working with is very large. Rather than
sending it via email, I have uploaded it to my crimson drive at this URL:
\url{https://drive.google.com/open?id=1YnHmRxBWaxg9Jqb6qEjJ8_2d9EYxY-72}.

I tested my project by encoding 147 videos at different quality levels. The
total size of the original videos was 266M, and the resulting output was
215M. This is an impressive file size reduction of 19\%, which comes at the cost
of only a mild reduction in video quality on the edges of the video.

These results could potentially be very useful for ultra-widescreen video where
the edges of the video are less important, such videos of the first person
perspective where the center of the video always contains the most important
information.

As it so happens, I started tracking what I do with my time last year, so I can
report with confidence that between last semester and the continuation this
semester, I have spent just over 220 hours on this course.

\end{document}
